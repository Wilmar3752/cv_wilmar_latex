\usepackage{forarray}


\newcommand{\EducationTemplate}[6]
{
\DefineArrayVar{#1}{@}
{|}{Start|End|School|Course|Stream}
{|}{{#2}|{#3}|{#4}|{#5}|{#6}}
}

\newcommand{\ExpTemplate}[7]
{
\DefineArrayVar{#1}{@}
{|}{Start|End|Designation|Company|Location|Highlights}
{|}{{#2}|{#3}|{#4}|{#5}|{#6}|{#7}}
}

\newcommand{\HckTemplate}[5]
{
\DefineArrayVar{#1}{@}
{|}{Date|Title|Position|Highlights}
{|}{{#2}|{#3}|{#4}|{#5}}
}

\def\sett#1#2#3{\expandafter\def\csname#1#2\endcsname{#3}}
% \def\appnd#1#2#3{\expandafter\def\csname#1#2\endcsname{#3}}
\usepackage{blindtext}  
\usepackage{lipsum}

\def\lrand{\lipsum[-]}
\def\firstname{Wilmar}
\def\lastname{Sepulveda Herrera}
\def\email{wilmar.sepulveda2@gmail.com}
\def\linkedin{https://www.linkedin.com/in/wilmar3752/}
\def\github{https://github.com/Wilmar3752}
\def\jasphone{+57 (311) 371-6605}

% \def\jbegin{\begin{itemize}}
% \def\jitem{\item }
% \def\jend{\end{itemize}}
\def\jbegin{\resumeItemListStart}
\def\jitem{\resumeItem}
\def\jend{\resumeItemListEnd}

\def\profileSummary{
\section{Sobre mi}
Estadístico MSc, con +5 años de experiencia desarrollando soluciones analíticas end to end basadas en modelos estadísticos, algoritmos de machine learning e inteligencia artificial 
para el sector financiero y asegurador, apasionado por el MLOPS, la ingeniería de datos, la docencia y la investigación. Durante mi carrera he logrado que los negocios de créditos y 
seguros para los que he trabajado sean mas rentables y tengan una mejor cultura de data.
}


\def\achievements{
    \jbegin
        \jitem{\lrand[1]}
        \jitem {\lrand[2]}
        \jitem {\lrand[1]}
        \jitem {\lrand[2]}
    \jend
}

% # Add education item here
\EducationTemplate{post}
{Agosto 2020}{Diciembre 2023}{Universidad del Valle}
{Magister en estadística}
{4.7/5}
\EducationTemplate{pre}
{Enero 2011}{Diciembre 2017}{Universidad del Valle}
{Estadístico}
{4/5}


\ExpTemplate{R52}
{Marzo 2022}
{Presente}
{Líder de Analítica Avanzada}
{\href{https://www.grupor5.com/}{R5}}
{Fintech}
{
\jbegin
    \jitem{Liderar un equipo 4 científicos de datos, trabajando en proyectos de Machine Learning, Analitica descriptiva y Mlops.}
    \jitem{Mediante Machine Learning y analitica descriptiva, logré mejorar el producto SOAT escogiendo inteligentemente los mejores riesgos y llegar a un LossRatio de 70\% vs un 120\% del mercado lo que se traduce en miles de millones de pesos de ganancias para la compañia.}
    \jitem{Usando Aprendizaje no supervisado, logramos entender mejor a nuestros clientes y así mejorar la eficiencia en las cobranzas.}
    \jitem{Con el uso de metodologías de analitica avanzada, logramos escoger mejor a nuestros clientes de credito, reduciendo asi metricas como 30-ever y 60-ever.}
\jend
}
\ExpTemplate{R51}
{Nov 2023}
{Dic 2023}
{Senior Data Scientist}
{\href{}{Alcaldíá de Santiago de Cali}}
{Contratista}
{
\jbegin
    \jitem{Implementar modelos de Machine learning para predecir mortalidad fetal y no fetal.}
    \jitem{Construcción de un tablero de control para cacterizar la mortalidad infantil.}
\jend
}

\ExpTemplate{Excelcredit}
{Mayo 2021}
{Marzo 2022}
{Senior Data Scientist}
{Excelcredit}
{Fintech}
{
\jbegin
    \jitem{Solucionar problemas de negocio mediante analítica avanzada, tales como: Calculo de provisión, estimación de prepago, duración de clientes, Segmentación de clientes, riesgo LAFT.}
\jend
}


\ExpTemplate{BDO}
{Sept 2020}
{Mayo 2021}
{Data Scientist}
{Banco de Occidente}
{Banca tradicional}
{
\jbegin
    \jitem{Encargado de construir modelos de machine learning para optimizar procesos y generar valor: Cobranza,  normalización de cartera,  daño en el flujo de caja de los clientes a raíz de la coyuntura del Covid 19.}
\jend
}

\ExpTemplate{UV}
{Abril 2019}
{Dic 2019}
{Data Analyst}
{Universidad del valle}
{Oficina de planeación}
{
\jbegin
    \jitem{Construcción de tableros de control para el análisis de las pruebas Saber-Pro, utilizando ShinyApps.}
    \jitem{Construcción de anuario estadístico 2018.}
    \jitem{Carga de diferentes indicadores de la universidad en el SNIES.}
\jend
}

\ExpTemplate{ulibre}
{Nov 2023}
{dic 2023}
{Docente}
{Universidad Libre - Cali}
{Doplomado Machine Learning}
{
\jbegin
    \jitem{Impartir el curso Estadística para Ciencia de Datos.}
\jend
}
\ExpTemplate{uvalle}
{Nov 2021}
{Feb 2022}
{Asistente de docencia}
{Universidad del valle}
{Facultad de Ingeniería}
{
\jbegin
    \jitem{Impartir el curso de Probabilidad y Estadística a estudiantes de Ingeniería de Sistemas.}
    \jitem{Impartir el curso de Fundamentos de Estadística a estudiantes de Ingenieríá Agrícola.}
\jend
}

\ExpTemplate{comfe}
{Febrero 2019}
{Enero 2021}
{Docente}
{Comfenalco - PEC}
{Tecnico laboral}
{
\jbegin
    \jitem{Orientar a los afiliados a la caja de compensación pertenecientes a los programas técnicos laborales; auxiliar administrativo y auxiliar contable (matemáticas, procesar datos, trabajo de grado.)}
\jend
}



\HckTemplate{HackCIMDGS}
{2019}
{\lrand[1]}
{\lrand[3]}
{\jbegin
    \jitem{\lrand[27]}
    \jitem{ \lrand[28]}
\jend}


\HckTemplate{HackWInfy}
{2019}
{\lrand[1]}
{\lrand[3]}
{\jbegin
    \jitem{\lrand[27]}
    \jitem{ \lrand[28]}
    \jitem{ \lrand[30]}
\jend}

\HckTemplate{i3wm}
{}
{\lrand[1]}
{\lrand[2]}
{\jbegin
    \item{\lrand[20]{}}
    \item {\lrand[30]}
    \item {\lrand[2]}
\jend}

\HckTemplate{scrapper}
{}
{\href{https://github.com/Wilmar3752/meli_scrapper}{Web Scraping MELI}}
{Realiza Web scraping de la lista de todos los productos publicados en Mercado Libre. Aquí certifico \\ mis conocimientos en Web Scraping, Docker, CI/CD, Python.}
{\jbegin
\jend}

\HckTemplate{scrapperETL}
{}
{\href{https://github.com/Wilmar3752/ETL_scraper}{CAR ETL}}
{Realizo un proceso de ETL donde extraigo datos de vehiculos, los proceso con Python y los cargo a AWS S3,\\ usando GitHub Actions.}
{\jbegin
\jend}

\HckTemplate{carpredict}
{}
{\href{https://github.com/Wilmar3752/car_predict}{CAR PREDICT}}
{Realizo un modelo de machine learning utilizando diferentes tecnologias: DVC, FastAPI, Docker, Python.}
{\jbegin
\jend}

\HckTemplate{carpredictapp}
{}
{\href{https://github.com/Wilmar3752/car_predict}{CAR PREDICT APP}}
{Aquí se realiza una aplicacion web con Streamlit, la cual tiene su respectiva base de datos.}
{\jbegin
\jend}

\HckTemplate{itseries}
{}
{\href{https://github.com/Wilmar3752/itseries}{ITSERIES}}
{Un paquete de R para analizar procesos estocásticos irregularmente espaciados.}
{\jbegin
\jend}

\HckTemplate{cluster}
{}
{\href{https://github.com/Wilmar3752/cluster-app}{Cluster-APP}}
{En Este proyecto realizo una APP con Gradio y HuggingFace para clasificar clientes de tarjeta de crédito.}
{\jbegin
\jend}
% \DefineArrayVar{Education}{@}
% {,}{School, Course, Stream, Start, End}
% {,}{C,D}


% % # Add experience item here




% \renewcommand{addEducation}{}


% # Add projects here
